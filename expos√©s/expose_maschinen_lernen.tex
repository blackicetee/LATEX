\documentclass[12pt,a4paper]{article}
\usepackage[utf8]{inputenc}
\usepackage[german]{babel}
\usepackage[T1]{fontenc}
\usepackage{makeidx}
%Mit dem Befehl \RM{Zahl in numerischem Format} werden jetzt Römische Zahlen erzeugt 
%\newcommand{\RM}[1]{\MakeUppercase{\romannumeral #1{.}}}


\author{Thilo Stegemann}
\title{Exposé: Entwicklung einer lernfähigen Maschine am Beispiel eines simplen mit ncurces programmierten Computerspiels}

\begin{document}
\maketitle

\section*{Motivation}
%Verstehen der Konzepte der Anfänge Spielentwicklung
%Angehende Entwickler und Neulinge auf dem Gebiet des maschinellen Lernens sollen die
Wie funktioniert eigentlich künstliche Intelligenz? Gibt es eine anschauliche und einfache Möglichkeit die Grundprinzipien des maschinellen Lernens zu verstehen? Ist das entwickeln von Computerspielen und künstlicher Intelligenz zu kompliziert?
Innerhalb dieser Bachelorarbeit wird versucht antworten auf diese Fragen zu finden. Die Grundlagen des maschinellen Lernens und der Spielentwicklung sollen erklärt und veranschaulicht werden. 

\section*{Vorläufige Zielsetzung}
Das Ziel der Arbeit ist die Entwicklung eines rundenbasierten 2-D Strategiespiels, wobei ein Mensch gegen eine künstliche Intelligenz(KI), also einen Computergegner, antritt. Diese KI soll sich den Strategien des Spielpartners anpassen können d.h. die KI soll lernen wie sie den menschlichen Mitspieler, innerhalb des Spiels, schlagen kann. Der menschliche Mitspieler wiederum soll vor eine strategische Herausforderung gestellt werden.

\section*{Methoden und Vorgehen}

\section*{Erwartbare Ergebnisse}
\begin{itemize}
	\item Ein lernfähiger Computergegner
	\item Rundenbasiertes Strategiespiel für eine Person gegen einen Computergegner 
\end{itemize}

\section*{Zeitplan}
\begin{itemize}
	\item Januar:
	\begin{itemize}
		\item \textbf{Literaturrecherche:} maschinelles Lernen, Entwicklung und Design von 2-D Spielen, Bibliothek ncurces
		\item \textbf{Schreiben der wissenschaftlichen Arbeit:} Grundlagen, Design und beginn der Kapitel Implementierung und Test
		\item \textbf{Implementierung und Test:} Algorithmen für maschinelles Lernen und Aufbau des Computerspiels
	\end{itemize}
	\item Februar:
	\begin{itemize}
		\item \textbf{Weiterhin ausführliches Studium} der ausgewählten Literatur
		\item \textbf{Schreiben der Kapitel}: Implementierung, Validierung, Abstrakt, Einleitung, Auswertung und Ausblick
		\item \textbf{Fertigstellen und Auswerten} der maschinellen Lernalgorithmen und des Computerspiels
	\end{itemize}
	\item März:
	\begin{itemize}
		\item \textbf{Feinschliff} der wissenschaftlichen Arbeit insbesondere Korrekturlesen
		\item \textbf{Pufferzeit} für etwaige Komplikationen
		\item \textbf{Drucken} der Arbeit
	\end{itemize}
\end{itemize}

\section*{Grobgliederung}
Abstrakt\\
Tabellenverzeichnis\\
Literaturverzeichnis
\begin{enumerate}
	\item Einleitung
	\begin{enumerate}
		\item Motivation
		\item Zielsetzung
		\item Stand der Technik
	\end{enumerate}
	\item Grundlagen
	\begin{enumerate}
		\item Maschinelles Lernen
		\item Grafik Bibliothek: Ncurces
	\end{enumerate}
	\item Design
	\begin{enumerate}
		\item Spielprinzip
		\item Design-Entwurf
		\item Strategien
	\end{enumerate}
	\item Implementierung
	\begin{enumerate}
		\item Belohnen und Bestrafen
		\item Mustererkennung Strategien
		\item Das Spiel 
	\end{enumerate}
	\item Validierung
	\begin{enumerate}
		\item Testen des maschinellen Lernens
		\item Testen des Spiels
		\item Auswertung der Tests
	\end{enumerate}
	\item Zusammenfassung
	\item Ausblick
	\item Literatur	
\end{enumerate}
Anhang

\section*{Literatur}


\end{document}