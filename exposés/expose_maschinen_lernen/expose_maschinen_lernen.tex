\documentclass[12pt,a4paper]{scrartcl}
\usepackage[utf8]{inputenc}
\usepackage{csquotes}
\usepackage[german]{babel}
\usepackage[T1]{fontenc}
\usepackage{makeidx}
\usepackage{graphicx}

\usepackage[backend=bibtex,
%style=numeric,
style=alphabetic,
%style=reading,
bibencoding=ascii]{biblatex}
\addbibresource{bibliography/bibliography}

\begin{document}
% Titleinformation formatted with KOMA-Script "scrartcl"
\subject{Exposé Bachelorarbeit}
\author{
	\texttt{Thilo Stegemann}\\
	\texttt{s0539757}\\
	\texttt{Angewandte Informatik}}
\title{Untersuchung der Lernfähigkeit einer künstlichen Intelligenz am Beispiel eines Computerspiels}

\maketitle
\centerline{\includegraphics[scale=1.5]{htw_logo}}

\section*{Motivation}
%Sind Sie ein (angehender) Softwareentwickler und programmieren aktuell ein Computerspiel mit Anwendungsbedarf von künstlicher Intelligenz? Benötigen Sie innerhalb einer beliebigen Anwendung eine lernfähige künstliche Softwarekomponente und wissen nicht welche Lernalgorithmen sich dafür eignen würden? 

%Innerhalb dieser Arbeit werden die Grundlagen des maschinellen Lernens(ML) bzw. der künstlichen Intelligenz(KI)  vermittelt.45 Zudem werden bestimmte Lernalgorithmen auf Ihre Funktionsweise, Belastbarkeit und Grenzen untersucht, implementiert, getestet und ausgewertet. Der Fokus innerhalb dieser Arbeit konzentriert sich auf die Untersuchung des Lernverhaltens einer KI.

Tritt jemand gegen einen Dame(Das Brettspiel "Dame") Computergegner an und verliert 20 Runden in Folge, dann kann das sehr frustrierend sein. Entweder man versucht verzweifelt die künstliche Intelligenz(KI) des Dame Computergegners immer wieder herauszufordern oder man entwickelt das dringende Verlange zu verstehen wie ein Dame Computergegner funktioniert. Ich persönlich wollte unbedingt verstehen wie das Lernverhalten und der Lernprozess der Dame-KI auf elementarer Ebene aussieht. Dabei interessieren mich besonders folgende Fragen:\\       

Wie lernt ein Programm Strategien? Was sind die elementaren Schritte die ein Programm  während des Lernprozesses durchläuft? Wie verändert sich das Lernverhalten einer künstlichen Intelligenz, wenn sie von verschiedenen menschlichen Gegnern trainiert wird? In wie fern wird die künstliche Intelligenz von einem Computerspiel ausgereizt? Wenn zwei Dame-KIs gegeneinander spielen und die Dame-KIs mit verschiedenen Lernalgorithmen implementiert sind, welche KI lernt dann effizienter, schneller oder besser?

\section*{Vorläufige Zielsetzung}
Das Ziel der Arbeit ist die Untersuchung des Lernverhaltens einer künstlichen Intelligenz am Beispiel eines eigens entworfenen zweidimensionalen rundenbasierten Strategiespiels(ZRS). Ein einfaches aber nicht triviales ZRS soll entworfen und implementiert werden. Dieses ZRS soll die Basis für die Untersuchung des Lernverhaltens der KI sein. Die KI wiederum soll maschinell lernen können und sich durch das Training gegen einen menschlichen Spielgegner verbessern. Speziell soll untersucht werden wie ein Algorithmus bzw. die KI, anhand von einem Computerspiel, lernt, was die Vor- und Nachteile verschiedener Lernalgorithmen sind und wo die Schwachstellen und Grenzen der KI/Lernalgorithmen liegen. 

\section*{Methoden und Vorgehen}
\begin{enumerate}
	\item Recherche und Studium relevanter Literatur
	
	\item Gliedern und herausbilden zielführender Passagen innerhalb der Literatur
	
	\item Einführung in das Thema
	\begin{enumerate}
		\item Klärung Motivation und Relevanz der Arbeit
		\item Stand der aktuellen Forschung aufzeigen
		\item Definition der Zielsetzung
		\item Darstellung der Ergebnisse
	\end{enumerate}		
	
	\item Vermitteln der Grundlagen für besseres Verständnis
	\begin{enumerate}
		\item Spielentwicklung
		\item Maschinelles Lernen und künstliche Intelligenz
	\end{enumerate}		
	
	\item Design des Spiels:
	\begin{enumerate}
		\item Entwurf des Spielprinzips
		\item Definieren der Spielregeln
		\item Entwerfen der Benutzerschnittstelle(User Interface)
		\item Design der Spielwelt 
		\item "Optional"[Verallgemeinern der Grundlagen und Implementierung eines Computerspiels]
		\item Festlegen der funktionalen Anforderungen
	\end{enumerate}

	\item Design der künstlichen Intelligenz:
	\begin{enumerate}
		\item Analyse möglicher Lernalgorithmen
		\item Kritische Bewertung: Wird der Lernalgorithmus vom ZRS ausgereizt?
		\item Erwartbares Lernverhalten der KI am Beispiel des ZRS
	\end{enumerate}
	
	\item Implementieren des ZRS und der Lernalgorithmen
	
	\item Alternative Implementierung und Algorithmen für ein anderes Lernverhalten der KI
	
	\item Testen bzw. Validieren der Funktionsweise des Computerspiels und des Lernverhaltens der KI
	\begin{enumerate}
		%TODO Wie testet man ein Computerspiel? Automatisches Testen?
		\item Messbare Kriterien für die Lernfähigkeit der KI entwerfen
		\item Empirisches Protokoll zum erfassen und untersuchen des Lernprozesses der KI mittels messbarer Kriterien
		\item Lernverhalten der KI im Hinblick auf wechselnde menschliche Spieler
		\item Ab wie viel wechselnden Spielern erreicht die KI Ihr Lernmaximum?
	\end{enumerate}
	
	\item Kritische Zusammenfassung der Untersuchung und der Ergebnisse
	
	\item Ausblick: Zukünftige Erweiterungen der Spielwelt, des Spielprinzips und der Lernalgorithmen
	\begin{enumerate}
		\item Ein anderes Spiel
		\item Neuronale Netze
	\end{enumerate}
	
\end{enumerate}

\section*{Erwartbare Ergebnisse}
\begin{itemize}
	\item Untersuchung des Lernverhaltens einer KI am Beispiel eines zweidimensionalen rundenbasierten Strategiespiels(ZRS)
	\item Ein ZRS für eine Person welche gegen einen lernfähigen Computergegner antreten kann 
\end{itemize}

\section*{Zeitplan}
\begin{itemize}
	\item Januar:
	\begin{itemize}
		\item \textbf{Literaturrecherche:} maschinelles Lernen, künstliche Intelligenz, Entwicklung und Design von 2-D Spielen
		\item \textbf{Schreiben der wissenschaftlichen Arbeit:} Kapitel Grundlagen, Design, Implementierung und Validierung
		\item \textbf{Implementierung und Test:} Algorithmen für maschinelles Lernen und Aufbau des Computerspiels
	\end{itemize}
	\item Februar:
	\begin{itemize}
		\item \textbf{Weiterhin ausführliches Studium} der ausgewählten Literatur
		\item \textbf{Schreiben der Kapitel}: Abstrakt, Einleitung, Auswertung und Ausblick
		\item \textbf{Fertigstellen und Auswerten} der implementierten maschinellen Lernalgorithmen und des Computerspiels
	\end{itemize}
	\item März:
	\begin{itemize}
		\item \textbf{Feinschliff} der wissenschaftlichen Arbeit insbesondere Korrekturlesen
		\item \textbf{Pufferzeit} für etwaige Komplikationen
		\item \textbf{Drucken} der Arbeit
	\end{itemize}
\end{itemize}

\section*{Grobgliederung}
Abstrakt\\
Tabellenverzeichnis\\
Literaturverzeichnis
\begin{enumerate}
	\item Einleitung
	\begin{enumerate}
		\item Motivation
		\item Stand der Technik
		\item Zielsetzung
	\end{enumerate}
	
	\item Grundlagen
	\begin{enumerate}
		\item Spielentwicklung
		\item Maschinelles Lernen
	\end{enumerate}
	
	\item Design des zweidimensionalen rundenbasierten Strategiespiels(ZRS)
	\begin{enumerate}
		\item Spielprinzip
		\item Spielregeln
		\item Benutzerschnittstelle
		\item Spielwelt
		\item Funktionale Anforderungen
	\end{enumerate}
	
	\item Design der künstlichen Intelligenz(KI)
	\begin{enumerate}
		\item Lernalgorithmen
		\item Voraussichtliche Grenzen der Lernalgorithmen
		\item Erwartbares Lernverhalten
	\end{enumerate}		
	
	\item Implementierung
	\begin{enumerate}
		\item Design ZRS
		\item Design KI
		\item Alternative Implementierung der KI
	\end{enumerate}
	
	\item Validierung
	\begin{enumerate}
		\item Messbare Testkriterien entwickeln
		\item Empirisches Protokoll zum Lernverhalten der KI
		\item Austesten der Belastbarkeit und herausfinden der Grenzen der KI
		\item Auswertung der Tests
	\end{enumerate}
	
	\item Kritische Zusammenfassung der Ergebnisse

	\item Ausblick
	\begin{enumerate}
		\item Unterschiedliche Spielgrundlagen
		\item Andere Lernalgorithmen oder Lerngrundlagen(Neuronale Netze)
	\end{enumerate}			
	
	\item Literatur	
\end{enumerate}
Anhang

%This command makes a citation of the literature named: "Alpaydm"
%\cite{Alpaydm}

%Command \nocite{*} makes command \printbibliography print the bib without citing at all
\nocite{*}
%Prints in the thesis mentioned literature
\printbibliography

\end{document}