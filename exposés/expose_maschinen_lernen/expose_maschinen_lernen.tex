\documentclass[12pt,a4paper]{scrartcl}
\usepackage[utf8]{inputenc}
\usepackage{csquotes}
\usepackage[german]{babel}
\usepackage[T1]{fontenc}
\usepackage{makeidx}
\usepackage{graphicx}

\usepackage[backend=bibtex,
%style=numeric,
style=alphabetic,
%style=reading,
bibencoding=ascii]{biblatex}
\addbibresource{bibliography/bibliography}

\begin{document}
% Titleinformation formatted with KOMA-Script "scrartcl"
\subject{Exposé Bachelorarbeit}
\author{
	\texttt{Thilo Stegemann}\\
	\texttt{s0539757}\\
	\texttt{Angewandte Informatik}}
\title{Untersuchung der Lernfähigkeit verschiedener Verfahren am Beispiel von Computerspielen}

\maketitle
\centerline{\includegraphics[scale=1.5]{htw_logo}}

\section*{Motivation}
Sind Sie ein (angehender) Softwareentwickler und programmieren aktuell ein Computerspiel, welches lernfähige Verfahren unterstützen soll? Benötigen Sie innerhalb einer beliebigen Anwendung einen lernfähigen Algorithmus und sie kennen die Schwächen, Stärken, Grenzen und Anwendungsgebiete der Lernverfahren nicht?\\

Haben Sie sich auch schon mal eine der nachfolgenden Fragen gestellt oder interessieren Sie diese Fragen generell?\\

Wie lernt ein Programm Strategien? Was sind die elementaren Schritte die ein Programm  während des Lernprozesses durchläuft? Wie anwendbar und leistungsfähig sind die Lernverfahren hinsichtlich verschiedener Spielgrundlagen? In wie fern wird ein Lernverfahren von einem Computerspiel ausgereizt? Wenn zwei unterschiedliche Lernverfahren untersucht und verglichen werden, welches Lernverfahren ist dann effizienter, schneller oder besser?\\

Diese wissenschaftliche Arbeit könnte dann sehr interessant für Sie sein. Innerhalb dieser Arbeit werden bestimmte Lernverfahren, am Beispiel verschiedener Computerspiele, auf Ihre Funktionsweise, Schwächen, Stärken und Grenzen untersucht, implementiert, und getestet. 

\section*{Vorläufige Zielsetzung}
Das Ziel der Arbeit ist die Untersuchung des Lernverhaltens, der Grenzen, der Schwächen und der Stärken verschiedener Lernverfahren am Beispiel von mindestens zwei eigens implementierten Computerspielen. Die Lernverfahren sollen trainiert werden und dadurch mehr oder weniger eigenständige Siegesstrategien und Spielzugmuster entwickeln. Die Lernverfahren könnten sich gegenseitig trainieren oder sie trainieren indem sie gegen einen Menschen spielen. Der Fokus der wissenschaftlichen Arbeit liegt hierbei auf der Untersuchung der verschiedenen Lernverfahren und nicht auf der Implementierung besonders komplexer Computerspiele, daher sollen nur sehr simple Computerspiele implementiert werden. Ein vollständiges Dame Spiel wird zum Beispiel nicht implementiert, aber eine absichtlich verkleinerte Dame Variante mit veränderten Spielregen, für ein schnelleres Spielende, wäre durchaus möglich. Zudem wären auch ein vier mal vier Tic-Tac-Toe ein Vier Gewinnt oder ein Black Jack Computerspiel möglich.

\section*{Methoden und Vorgehen}

\begin{itemize}
	\item Implementierung der Computerspiele
	\item Implementierung der lernfähigen Verfahren, am Beispiel der Computerspiele, mit Unterstützung von Bibliotheken
	\item Validieren und Auswerten der lernfähigen Verfahren am Beispiel der Computerspiele
	\begin{itemize}
		\item Empirisches Programmprotokoll, soll den Lernprozess der Verfahren auf elementarer Ebene festhalten
		\item Untersuchung der Schwächen, Stärken, Grenzen und der optimalen Auslastung der Lernverfahren\\\\
	\end{itemize}
\end{itemize}	

\section*{Erwartbare Ergebnisse}
\begin{itemize}
	\item Mindestens zwei implementierte Computerspiele
	\item Implementierung und Anwendung verschiedener Lernverfahren am Beispiel der implementierten Computerspiele
	\item Untersuchung der Schwächen, Stärken, Grenzen und Verwendbarkeiten der Lernverfahren am Beispiel der Computerspiele
\end{itemize}

\section*{Zeitplan}
\begin{itemize}
	\item Januar:
	\begin{itemize}
		\item \textbf{Literaturrecherche:} Maschinelles Lernen, künstliche Intelligenz, Entwicklung und Design von 2-D Spielen
		\item \textbf{Schreiben der wissenschaftlichen Arbeit:} Kapitel Grundlagen, Anforderungen und Konzepte, Implementierung und Validierung
		\item \textbf{Implementierung und Test:} Der Lernverfahren und der Computerspiele
	\end{itemize}
	\item Februar:
	\begin{itemize}
		\item \textbf{Weiterhin ausführliches Studium:} Der ausgewählten Literatur
		\item \textbf{Schreiben der Kapitel:} Abstrakt, Einleitung, Auswertung und Ausblick
		\item \textbf{Fertigstellen und Auswerten:} Der Lernverfahren und der Computerspiele
	\end{itemize}
	\item März:
	\begin{itemize}
		\item \textbf{Feinschliff:} Der wissenschaftlichen Arbeit insbesondere Korrekturlesen
		\item \textbf{Pufferzeit:} Für etwaige Komplikationen
		\item \textbf{Drucken:} Der wissenschaftlichen Arbeit
	\end{itemize}
\end{itemize}

\section*{Grobgliederung}
Abstrakt\\
Abbildungsverzeichnis
Tabellenverzeichnis
\begin{enumerate}
	\item Einleitung
	\begin{enumerate}
		\item Motivation
		\item Zielsetzung
	\end{enumerate}
	
	\item Grundlagen
	\begin{enumerate}
		\item Spielentwicklung
		\item Lernfähige Verfahren
	\end{enumerate}

	\item Anforderungen und Konzepte
	\begin{enumerate}
		\item Computerspiele
		\begin{enumerate}
			\item Spielprinzipien
			\item Spielregeln
			\item Benutzerschnittstellen
			\item Formale Anforderungen
		\end{enumerate}
		\item Lernverfahren
		\begin{enumerate}
			\item Strategie und Auswahl
			\item Spiele als Anwendungsgrundlage
			\item Konzeptuelles Training
			\item Persistenz der Trainingsdaten
		\end{enumerate}
	\end{enumerate}		
	
	\item Implementierung
	\begin{enumerate}
		\item Spiele
		\item Lernverfahren
		\item Alternative Lernverfahren
	\end{enumerate}
	
	\item Validierung
	\begin{enumerate}
		\item Messbare Testkriterien entwickeln
		\item Empirisches Protokoll zum Lernverhalten der Verfahren
		\item Austesten der Belastbarkeit und herausfinden der Grenzen der Verfahren
		\item Untersuchung der Verwendbarkeit der Lernverfahren 
	\end{enumerate}
	
	\item Kritische Zusammenfassung der Ergebnisse

	\item Ausblick
	\begin{enumerate}
		\item Andere Computerspiele als Anwendungsgrundlage
		\item Weitere interessante Lernverfahren(Neuronale Netze)
	\end{enumerate}			
	
	\item Literatur	
\end{enumerate}
Anhang

%This command makes a citation of the literature named: "Alpaydm"
%\cite{Alpaydm}

%Command \nocite{*} makes command \printbibliography print the bib without citing at all
\nocite{*}
%Prints in the thesis mentioned literature
\printbibliography

\end{document}