% I want to use UTF8 always
\usepackage[utf8]{inputenc}

% the language of the thesis (babel recommends loading csquotes)
\usepackage{csquotes} 
\usepackage[german]{babel}

% These two packages are needed for math and math symbols, which I use
% heavily in the document.
\usepackage[fleqn]{amsmath}
\usepackage{amssymb}

% siunitx is great for typesetting units like a = 5mm. I try to use it
% whenever any unit is used. 
\usepackage{siunitx}

% fonts. Pallatino, basically.
\usepackage[sc]{mathpazo}
\usepackage[scaled=0.86]{berasans}
\usepackage[scaled=1.03]{inconsolata}
\usepackage{tgpagella}

% use a better font encoding. 7 bit are NOT enough!
\usepackage[T1]{fontenc}

% This package lets me use \includegraphics for figures.
\usepackage{graphicx}

% This package is for highlighting notes especially todos
\usepackage{todonotes}

% nicer tables
%\usepackage{booktabs}
\usepackage{tabularx}

% tikz graphics
\usepackage{tikz}

% Make the table of contents be linked to the content position.
\usepackage{hyperref}
\definecolor{darkblue}{rgb}{0,0,0}
\hypersetup{
%    pdftex=true, 
    colorlinks=true, 
    breaklinks=true, 
    linkcolor=darkblue, 
    menucolor=darkblue, 
%    pagecolor=darkblue, 
    urlcolor=darkblue,
    citecolor = darkblue
}

% References with biber. 
%\usepackage[
%    backend = biber,
%    firstinits = true,
%    %style = phys, % the phys style must be installed separately
%    sortlocale = en_US,
%    natbib = true,
%    doi = false,
%    eprint = false,
%    sorting=none
%]{biblatex}

\usepackage[
	backend=bibtex,
	%style=numeric,
	style=alphabetic,
	%style=reading,
	bibencoding=ascii
]{biblatex}

% clever references
\usepackage[noabbrev, capitalise]{cleveref}

% fix latex2e bugs
\usepackage{fixltx2e}

% line numbers
\usepackage[modulo,switch]{lineno}
%\linenumbers

% how is our text width?
\usepackage{printlen}

% include pdf documents
\usepackage{pdfpages}

% list of abbreviations
\usepackage{acronym}

% chemical formulas
%\usepackage[version=3]{mhchem}