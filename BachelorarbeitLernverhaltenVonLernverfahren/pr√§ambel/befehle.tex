% These are the actual shortcuts I used in my thesis. They might give 
% you an idea on how to define your own. I highly recommend not to repeate 
% yourself but rather write makros for stuff you often use!

% The default exponential function sucks a bit, so let's define our own
% ones.
\newcommand{\eexp}[1]{\exp\!\left(#1\right)}
\newcommand{\eeexp}[1]{\exp\!\left[#1\right]}
\newcommand{\eeeexp}[1]{\exp\!\left\{#1\right\}}

% shortcuts
\newcommand{\ei}[0]{\varepsilon_\infty}
\newcommand{\ef}[0]{\varepsilon_F}
\newcommand{\et}[0]{\varepsilon_t}
\newcommand{\enot}[0]{\varepsilon_0}
\newcommand{\ve}[0]{\varepsilon}

\newcommand{\placeholder}[0]{
        \framebox[\textwidth]{
                \rule{0pt}{5cm}
        }
}

\newcommand{\avg}[1]{\langle #1 \rangle}

\newcommand{\blankpage}[0]{
        \newpage
        \thispagestyle{empty}
        \mbox{}
}

% Command will make a linebreak after the head of a paragraph
\newcommand{\myparagraph}[1]{\paragraph{#1}\mbox{}\\}

\newcommand{\tab}{$\-$ $\-$ $\-$ $\-$ $\-$ $\-$ $\-$ $\-$ $\-$ $\-$}


% Sectio below defines math operations and functions which are not supported by build in latex.
\DeclareMathOperator*{\argmax}{arg\!\max}