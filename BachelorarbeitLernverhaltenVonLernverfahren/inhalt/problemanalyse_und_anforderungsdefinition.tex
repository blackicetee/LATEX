\chapter{Problemanalyse und Anforderungsdefinition}
\label{cha:pua}

In diesem Kapitel: //TODO Einführung in das Kapitel

\section{Problemanalyse}

\subsection{Die Problematik}

\subsection{Bereits existierende Softwarelösungen}

\subsection{Gegenstandsbereich des Projektes abgrenzen}

\subsection{Abgrenzung gegenüber der künstlichen Intelligenz}

\section{Anforderungsdefinition}
Innerhalb dieses Abschnittes werden die funktionalen und nicht-funktionalen Anforderungen für die Software definiert. Die funktionalen Anforderungen sollen das Verhalten der Software festlegen, sprich welche Aktionen die Software ausführen kann. In verschiedenen Softwareprojekten können sich die funktionale Anforderungen stark voneinander unterscheiden, dahingegen sind nicht-funktionale Anforderungen in verschiedenen Softwareprojekten oftmals sehr ähnlich.\\ 
Nichtfunktionale Anforderungen beschreiben wie gut eine Software seine Leistung erbringen soll.
%TODO definition, Belegung in der Literatur dafür geben Softwareentwicklung
Die funktionalen und nichtfunktionalen Anforderungen sollen Muss- und Soll-Kriterien beinhalten. Muss-Kriterien beschreiben die Kernfunktionalitäten der Software und ohne diese ist eine Problemstellung nicht zufriedenstellend zu lösen. Soll-Kriterien beschreiben nicht die Kernfunktionalitäten, jedoch können sie die Kernfunktionalitäten erweitern und nützliche Zusatzfunktionalitäten definieren.

\subsection{Funktionale Anforderungen}

\begin{tabular}{lcr}
Nummer & Titel & Beschreibung
\end{tabular}

\subsection{Nicht-funktionale Anforderungen}


