\chapter{Projektvision}
\label{cha:projektvision}

%TODO Was passiert in diesem Kapitel?

\section{Motivation}
Sind Sie ein (angehender) Softwareentwickler und programmieren aktuell ein Computerspiel, welches lernfähige Verfahren unterstützen soll? Benötigen Sie innerhalb einer beliebigen Anwendung einen lernfähigen Algorithmus und sie kennen die Schwächen, Stärken, Grenzen und Anwendungsgebiete der Lernverfahren nicht?\\

Haben Sie sich auch schon mal eine der nachfolgenden Fragen gestellt oder interessieren Sie diese Fragen generell?\\

Wie lernt ein Programm Strategien? Was sind die elementaren Schritte die ein Programm  während des Lernprozesses durchläuft? Wie anwendbar und leistungsfähig sind die Lernverfahren hinsichtlich verschiedener Spielgrundlagen? In wie fern wird ein Lernverfahren von einem Computerspiel ausgereizt? Wenn zwei unterschiedliche Lernverfahren untersucht und verglichen werden, welches Lernverfahren ist dann effizienter, schneller oder besser?\\

Diese wissenschaftliche Arbeit könnte dann sehr interessant für Sie sein. Innerhalb dieser Arbeit werden bestimmte Lernverfahren, am Beispiel verschiedener Computerspiele, auf Ihre Funktionsweise, Schwächen, Stärken und Grenzen untersucht, implementiert, und getestet. 

\section{Vorläufige Zielsetzung}
Das Ziel der Arbeit ist die Untersuchung des Lernverhaltens, der Grenzen, der Schwächen und der Stärken verschiedener Lernverfahren am Beispiel von mindestens zwei eigens implementierten Computerspielen. Die Lernverfahren sollen trainiert werden und dadurch mehr oder weniger eigenständige Siegesstrategien und Spielzugmuster entwickeln. Die Lernverfahren könnten sich gegenseitig trainieren oder sie trainieren indem sie gegen einen Menschen spielen. Der Fokus der wissenschaftlichen Arbeit liegt hierbei auf der Untersuchung der verschiedenen Lernverfahren und nicht auf der Implementierung besonders komplexer Computerspiele, daher sollen nur sehr simple Computerspiele implementiert werden. Ein vollständiges Dame Spiel wird zum Beispiel nicht implementiert, aber eine absichtlich verkleinerte Dame Variante mit veränderten Spielregen, für ein schnelleres Spielende, wäre durchaus möglich. Zudem wären auch ein vier mal vier Tic-Tac-Toe ein Vier Gewinnt oder ein Black Jack Computerspiel

\section{Nutzen/Zweck des Projektes}

\section{Aufbau der Arbeit}