\chapter{Grundlagen}
\label{ch:Grundlagen}

\todo[inline]{Schreiben der Einführung des Kapitels \ref{cha:grundlagen} Grundlagen}

\section{Lernfähige Verfahren}

\subsection{Die drei Arten des maschinellen Lernens}
\label{subsec:Die drei Arten des maschinellen Lernens}

\myparagraph{Überwachtes Lernen}

Verfahren aus der Gattung des überwachten Lernens,  benötigt Datensets mit bestimmten Eigenschaften. Ein überwachter lernfähiger Algorithmus zur Klassifizierung von Daten, benötigt ein Trainingsset bestehend aus Eigenschaften(Features) und den dazugehörigen Klassen bzw. Zielwerten(Target Values). Abbildung \missingfigure{Erstellen eines Trainingssets für überwachte Klassifizierung} stellt ein Beispiel für ein mögliches Trainingsset dar. Ziel der Klassifizierung ist das vorhersagen von Zielwerten. Die Qualität der Zielwerte kann mittels Testsets ermittelt werden. Ein Testset ist ein Datenset bestehend aus Eigenschaften ohne die dazugehörigen Klassen.

\myparagraph{Unüberwachtes Lernen}

\myparagraph{Bestärkendes Lernen}
Bestärkendes Lernen oder verstärkendes Lernen(engl. Reinforcement Learing) ist 

\section{Spielentwicklung}

\section{Lineare Algebra}

\section{Heuristik}