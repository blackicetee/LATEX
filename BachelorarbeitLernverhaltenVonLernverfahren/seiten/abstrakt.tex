\chapter*{Abstrakt}

Die vorliegende Bachelorarbeit gibt einen Überblick über Theorien des verstärkenden Lernens. Das TD-Q-Lernen ist ein verstärkendes Lernverfahren. 
In dieser Arbeit wird das TD-Q-Lernen, mit einer tabellarischen Repräsentation der Q-Funktion, erläutert, implementiert, getestet und beurteilt. 
Ein Agent der das TD-Q-Lernen anwendet, wird Strategien für die Strategiespiele Tic Tac Toe und Reversi lernen. 
Ein Zufallsagent und ein vorausschauender Heuristik Agent werden ebenfalls implementiert. Die Implementierung des vorausschauenden Heuristik Agenten wird eine Kombination aus einer Iterativen-Alpha-Beta-Suche und einer Heuristik. \\

In Testspielen wird der selbstlernende TD-Q-Agent gegen den Zufallsagenten und den vorausschauenden Heuristik Agenten antreten. Die Auswertung der Testergebnisse gibt Aufschluss über die Leistungsfähigkeit und die Grenzen des TD-Q-Lernens. 
Der Heuristik Agent wird in den Testspielen den TD-Q-Agenten eindeutig besiegen. Das bestätigt, warum in der Praxis überwiegend von Menschen optimierte Heuristiken und keine Lernverfahren für das Spielen von Strategiespielturnieren eingesetzte werden. 
Erkenntnis dieser Arbeit u.a. ist: Die tabellarische Darstellung der Q-Funktion für das TD-Q-Lernen ist ausschließlich erfolgreich für sehr kleine Zustands- und Aktionsräume anwendbar. Für größer dimensionierte Zustands- und Aktionsräume ist die parametrisierte Funktionsdarstellung der Q-Funktion erforderlich. 




